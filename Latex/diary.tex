\documentclass{article}
\usepackage[utf8]{inputenc}
\usepackage{hyperref}
\title{LegendOfBluespec}
\author{kr469 }
\date{October 2021}


\begin{document}

\maketitle

\section{11 / 1 / 2021}
    I found a bug in bluespec reference guide
    \hyperlink{https://github.com/BSVLang/Main/blob/master/Language_Spec/bsv-reference-guide.pdf}{Official pdf}
    On page 30 method proto has $([ methodProtoFormals ])$ 
    and it should have $[( methodProtoFormals)]$ 
    otherwise you can't declare method without any round brackets like it's done in IntStack example bellow
    as theirs nontaion assume $[]$ as zero or more. 

    \subsection{Lark grammar}
         I implemented grammar for parsing interfaces declatations written in bluespec (without support for subinteraces)
         and parsing for Interfaces results of command Bluetcl type full \( Name of interface \)

\section{11/ 2 / 2021}
    I added grammar for parsing functions creating modules this way one can 
    read what modules can be created how they need to be initiazlied and what interfaces they have.  

\section{11 / 3 / 2021}
    I added more text to main.tex about toolchains and a more detailed plan for structure of the library. I also added a openALL.bat file, that is for me to make opening useful pdfs easier.

\section{11/ 3-10 /2021}
    I worked on grammar and now grammar is capable of parsing almost all features of bluespec that are exposed trough defs type and defs funcs 
\section{11/ 11 / 2021}
    I'm capable of paring Interfaces (without subinterfaces), types and few other things like enums, there are some things left that I'm currently not parsing but I decided to move them for later as they are not currently needed and I'm not sure if I'll need them, I will prioritize for now support for JSON and generating bluespec.

\section{11/ 17 / 2021}
    I added simple bluespec generation and worked on integrating it with the toolchain. First bsv file was produced.

\end{document}
